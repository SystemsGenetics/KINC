\documentclass[10pt]{article}
\usepackage[T1]{fontenc}
\usepackage{tabularx}
\usepackage{parskip}
\usepackage{float}
\usepackage{hyperref}
\usepackage{amsmath}
\usepackage{xcolor}
\usepackage{graphicx}

\hypersetup{
    colorlinks,
    linkcolor={red!50!black},
    citecolor={blue!50!black},
    urlcolor={blue!80!black}
}

\setlength{\parindent}{0cm}

\begin{document}

\title{\textbf{KINC Specification}\\
\vspace{1mm}
\includegraphics[width=12cm,height=1cm]{dna}
\\Knowledge Independent Network Construction}
\author{Joshua Burns\\Stephen Ficklin}
\maketitle

\tableofcontents

\newpage
\section{Introduction}

This software is designed to take biological information and transform it into 
new types of data based off statistical analysis.

\newpage
\section{Abstract Classes}

These interface classes describe the overall structure of the kinc program and 
how implemenation classes work with one another.

The structure of this code will allow for user implemented plugins in the 
future, since the interaction of the interface classes are clearly defined.

\subsection{BioData}

This is responsible to storing and representing a certain type of biological 
data. It is responsible for reading and writing to a file the data it holds, 
merging two or more files of its type together into one, modifying the data it 
holds, and providing results to queries about the data it holds.

These classes are also responsible for importing from human readable file 
formats to their binary file format and exporting back out to human readable 
formats.

\subsection{BioAnalytic}

This is responsible for taking in one or more BioData objects and through 
statistical analysis producing one or more new BioData objects.

\newpage
\section{BioData Classes}

\subsection{Expression}

This is responsible for storing expression sample data for genes.

\subsubsection{File Structure}

\begin{list}{}{}
\item[1.] Special Identifier that identifies this file as an expression file.
\item[2.] User defined name for expression data.
\item[3.] History of this data, where it came from.
\item[4.] Number of genes, then number of samples per gene.
\item[5.] List of all genes.
\item[6.] 2 Dimensional list of all gene samples, per gene.
\end{list}

\subsection{Correlation}

This is responsible for storing correlation data between genes.

\subsubsection{File Structure}

\begin{list}{}{}
\item[1.] Special Identifier that identifies this file as a correlation file.
\item[2.] User defined name for correlation data.
\item[3.] History of this data, where it came from.
\item[4.] Number of genes, number of correlations per gene pair.
\item[5.] List of correlation methods used.
\item[5.] List of all genes.
\item[6.] Special lists of all gene correlations sorted by their correlation 
value. Each entry will have a correlation value, then a number representing how 
many gene correlations are this value, then the list of gene correlations. For 
each type of correlation used a list is made.
\item[7.] Special 2 Dimensional list of all gene correlations, special being a 
diagonal matrix instead of square.
\end{list}

\subsection{Network}

This is responsible for storing network data between genes.

\subsubsection{File Structure}

\begin{list}{}{}
\item[1.] Special Identifier that identifies this file as a network file.
\item[2.] User defined name for network data.
\item[3.] History of this data, where it came from.
\item[4.] Number of genes.
\item[5.] List of all genes.
\item[6.] Special 2 Dimensional list of all gene edges, special being a 
diagonal matrix instead of square.
\end{list}

\subsection{Annotation}

This is responsible for storing additional information for genes.

\subsubsection{File Structure}

\begin{list}{}{}
\item[1.] Special Identifier that identifies this file as an annotation file.
\item[2.] User defined name for annotation data.
\item[4.] Number of genes, number of annotations.
\item[5.] List of all genes.
\item[6.] List of all annotation types.
\item[7.] 2 Dimensional list of all gene annotations, per gene.
\end{list}

\newpage
\section{BioAnalytic Classes}

\subsection{Pearson}

This takes an Expression BioData object and produces a Correlation BioData 
object. It uses the Pearson correlation statistical method for giving 
correlation values.

\subsection{Spearman}

This takes an Expression BioData object and produces a Correlation BioData 
object. It uses the Spearman correlation statistical method for giving 
correlation values.

\end{document}
